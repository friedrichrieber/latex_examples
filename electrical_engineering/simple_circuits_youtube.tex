\documentclass{article}

\usepackage[utf8]{inputenc}
\usepackage[siunitx]{circuitikz} % siunitx damit man sachen beschriften kann 
\usepackage{hyperref}
\pagestyle{empty}
\begin{document}

nach diesem YouTube Video: \href{https://www.youtube.com/watch?v=WRTELZP1l0Y}{Link}\\

\begin{center}
  \begin{circuitikz} \draw
  (0,0) to[battery] (0,4)
  to[ammeter] (4,4) -- (4,0)
  to[lamp] (0,0)
  ;
  \end{circuitikz}
\end{center}

\begin{center}
  \begin{circuitikz} \draw
  (0,0) to[battery] (0,4)
  to[ammeter] (4,4) -- (4,0)
  to[lamp] (0,0)
  (0.5,0) -- (0.5, -2)
  to[voltmeter] (3.5, -2) -- (3.5, 0)
  ;
  \end{circuitikz}
\end{center}

\begin{center}
  \begin{circuitikz} \draw
  (0,0) to[battery] (0,4)
  to[ammeter] (4,4)
  to[C] (4,0) -- (3.5,0) % C = Capacitor
  to[lamp, *-*] (0.5,0) -- (0,0)
  (0.5,0) -- (0.5, -2)
  to[voltmeter] (3.5, -2) -- (3.5, 0)
  ;
  \end{circuitikz}
\end{center}

\begin{center}
  \begin{circuitikz} \draw
  (0,0) to[battery] (0,4)
  to[ammeter, l=2<\milli\ampere>] (4,4)
  to[C] (4,0) -- (3.5,0) % C = Capacitor
  to[lamp, *-*] (0.5,0) -- (0,0)
  (0.5,0) -- (0.5, -2)
  to[voltmeter] (3.5, -2) -- (3.5, 0)
  ;
  \end{circuitikz}
\end{center}


\begin{center}
  \begin{circuitikz} \draw
  (0,0) to[battery] (0,4)
  to[ammeter, l_=2<\milli\ampere>] (4,4) % der unterstrich gibt an ob die
                                % beschrifung oben oder unten ist 
  to[C] (4,0) -- (3.5,0) % C = Capacitor
  to[lamp, *-*] (0.5,0) -- (0,0)
  (0.5,0) -- (0.5, -2)
  to[voltmeter] (3.5, -2) -- (3.5, 0)
  ;
  \end{circuitikz}
\end{center}

\begin{center}
  \begin{circuitikz} \draw
  (0,0) to[battery] (0,4)
  to[ammeter, i_=2<\milli\ampere>] (4,4) % der unterstrich gibt an ob die
                                % beschrifung oben oder unten ist 
  to[C] (4,0) -- (3.5,0) % C = Capacitor
  to[lamp, *-*] (0.5,0) -- (0,0)
  (0.5,0) -- (0.5, -2)
  to[voltmeter] (3.5, -2) -- (3.5, 0)
  ;
  \end{circuitikz}
\end{center}

\begin{center}
  \begin{circuitikz} \draw
  (0,0) to[battery] (0,4)
  to[ammeter, i_=2<\milli\ampere>] (4,4) % der unterstrich gibt an ob die
                                % beschrifung oben oder unten ist 
  to[C=3<\farad>] (4,0) -- (3.5,0) % C = Capacitor, hier kann man die
                                % Beschriftung einfach danach hinschreiben
  to[lamp, *-*] (0.5,0) -- (0,0)
  (0.5,0) -- (0.5, -2)
  to[voltmeter, l=3<\kilo\volt>] (3.5, -2) -- (3.5, 0)
  ;
  \end{circuitikz}
\end{center}

\begin{center}
  \begin{circuitikz} \draw
  (0,0) to[battery] (0,4)
  to[ammeter, i_=2<\milli\ampere>] (4,4) % der unterstrich gibt an ob die
                                % beschrifung oben oder unten ist 
  to[C=3<\farad>] (4,0) -- (3.5,0) % C = Capacitor, hier kann man die
                                % Beschriftung einfach danach hinschreiben
  to[lamp, *-*] (0.5,0) -- (0,0)
  (0.5,0) -- (0.5, -2)
  to[voltmeter, l=3<\kilo\volt>, color=red] (3.5, -2) -- (3.5, 0)
  ;
  \end{circuitikz}
\end{center}

\begin{center}
  \begin{circuitikz}[scale=2]\draw %hier kann man die größe ändern 
  (0,0) to[battery] (0,4)
  to[ammeter, i_=2<\milli\ampere>] (4,4) % der unterstrich gibt an ob die
                                % beschrifung oben oder unten ist 
  to[C=3<\farad>] (4,0) -- (3.5,0) % C = Capacitor, hier kann man die
                                % Beschriftung einfach danach hinschreiben
  to[lamp, *-*] (0.5,0) -- (0,0)
  (0.5,0) -- (0.5, -2)
  to[voltmeter, l=3<\kilo\volt>, color=red] (3.5, -2) -- (3.5, 0)
  ;
  \end{circuitikz}
\end{center}

\begin{center}
  \begin{circuitikz} \draw
    (0,0) to[R, o-o] (2,0)
    (4,0) to[vR, o-o] (6,0)
    (0,2) to[tranmission line, o-o] (2,2)
    (4,2) to[closing switch, o-o] (6,2)
    (0,4) to[european current source, o-o] (2,4)
    (4,4) to[european voltage source, o-o] (6,4)
    (0,6) to[empty diode, o-o] (2,6)
    (4,6) to[full led, o-o] (6,6)
    (0,8) to[generic, o-o] (2,8)
    (4,8) to[sinusoidal voltage source, o-o] (6,8)
    ;
  \end{circuitikz}
\end{center}
\newpage
Wenn ein Komponent nicht ganz auf eine Linie past müssen sogenannte
\textit{Nodes} verwendet werden
\begin{center}
  \begin{circuitikz} \draw
    (0,0) node[antenna] {}
    (4,0) node[pmos] {}
    (0,4) node[op amp] {}
    (4,4) node[american or port] {}
    (0,8) node[transformer] {}
    (4,8) node[spdt] {}
    ;
  \end{circuitikz}
\end{center}


\end{document}